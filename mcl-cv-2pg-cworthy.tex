%!TEX program = pdflatex
\documentclass[12pt]{article}
\usepackage[left=2.54cm,top=2.54cm,right=2.54cm,bottom=2.54cm]{geometry}
\usepackage[inline]{enumitem}
\usepackage{palatino}
\usepackage{url}
\urlstyle{rm}

\usepackage{titlesec}
\titleformat{\section}
{\normalsize\bfseries}{\thesection}{1em}{}
\titlespacing\section{0pt}{3pt plus 2pt minus 2pt}{3pt plus 2pt minus 2pt}
\renewcommand{\thesection}{(\alph{section})}


\begin{document}
\thispagestyle{empty}

\begin{center}

\
\vspace{-2em}

\noindent
\textbf{Biographical sketch}

\textbf{Matthew C. Long}

\noindent
National Center for Atmospheric Research\\
P.O. Box 3000, Boulder, CO 80307\\
Phone: 303-497-1311; e-mail: mclong@ucar.edu\\
ORCID: 0000-0003-1273-2957
\end{center}

\vspace{-0.5em}
\section{Professional preparation}

\begin{tabular}{lllll}
Tufts University	&	Medford, MA 	& Environmental Engineering	& B.S.	& 1998 	\\
Tufts University	&  	Medford, MA		& Environmental Engineering	& M.S.	& 2000	\\
Stanford University	&	Stanford, CA	& Oceanography				& Ph.D.	& 2010 \\
NCAR & Boulder, CO & {Advanced Study Program} & Postdoc & 2010-12\\
\end{tabular}


\section{Appointments}

\begin{description}[style=multiline,leftmargin=2.8cm,font=\normalfont]
\setlength{\itemsep}{-0.3em}
\item[2022--present] {Director}; [C]worthy Project at Convergent Research.
\item[2014--present] {Scientist I, II, III}; Oceanography Section, Climate \& Global Dynamics \\ Laboratory, National Center for Atmospheric Research.
\item[2012--2014] {Project Scientist}; Oceanography Section, Climate \& Global Dynamics \\ Laboratory, National Center for Atmospheric Research.
\item[2005--2010] Research Assistant, Stanford University.
\item[2004--2009] Teaching Assistant, Stanford University.
\item[2003--2004] {Water Resources Engineer}; {Camp Dresser \& McKee Inc.}, Cambridge, MA.
\item[2000--2002] {High School Physics \& Geography Teacher}; US Peace Corps, Tanzania.
\item[1999--1999] Environmental Analyst, Massachusetts Department of Public Health.
\end{description}


\section{Selected publications
{\footnotesize ($^*$student led; $^\dagger$postdoc led)}}

\begin{enumerate}[leftmargin=1.5em,font=\normalfont]
\setlength{\itemsep}{-0.2em}

\item
\textbf{{Long}, M.~C.}, B. B. Stephens, K. McKain, C. Sweeny, R. Keeling, E. A. Kort, et al.
%E. Morgan, J. D. Bent, N. Chandra, F. Chevallier, R. Commane, B. Daube, P. B. Krummel, Z. Loh, I. T. Luijkx, D. Munro, P. Patra, W. Peters, M. Ramonet, C. Rödenbeck, A. Stavert, P. Tans, and S. C Wofsy
(2021), Strong Southern Ocean carbon uptake evident in airborne observations, \textit{Science},
\textbf{374}(6572), {1275-1280}.

\item
\textbf{{Long}, M.~C.}, Moore, J. K., Lindsay, K., Levy, M., Doney, S. C., Luo, J. Y., et al. (2021). Simulations with the Marine Biogeochemistry Library (MARBL). \textit{JAMES}, \textbf{13}, \\e2021MS002647.

\item
\textbf{{Long}, M.~C.}, T. Ito, and C. Deutsch (2019), Oxygen projections for the future, in Ocean deoxygenation:
everyone’s problem. Causes, impacts, consequences and solutions., edited by D. Laffoley and J. Baxter, doi:10.2305/IUCN.CH.2019.13.en.

\item
Ito, T., \textbf{M.~C. Long}, C.~Deutsch, S.~Minobe, D.~Sun (2019), Mechanisms of low-frequency O$_2$ variability in the North Pacific, \textit{Global Biogeochem. Cycles}, \textbf{33}(2), 110--124.

\item
$^\dagger$Harrison, C., \textbf{M.~C. Long}, N.~{Lovenduski}, J. K. Moore (2018), {Mesoscale effects on carbon export: a global perspective}. \textit{Global Biogeochem. Cycles}, \textbf{32}(4), 680--703.

\item
Moore, J.~K., W.~Fu, F.~Primeau, G.~L. Britten, K.~Lindsay, \textbf{M.~C. Long},
%et al.
S.~C.Doney, N.~Mahowald, F.~Hoffman, J.~T. Randerson
(2018), Sustained climate warming drives declining marine biological productivity, \textit{Science}, \textbf{359}(6380), 1139--1143.

\item
$^*${Krumhardt}, K.~M., N.~S. Lovenduski, \textbf{M.~C. Long}, and K.~Lindsay (2017),
  Avoidable impacts of ocean warming on marine primary production: Insights
  from the CESM ensembles, \textit{Global Biogeochem. Cycles},
  \textbf{31}(1), 114--133.

\item
Ito, T., S.~Minobe, \textbf{M.~C. Long}, C.~Deutsch (2017), {Upper Ocean O$_2$ trends: 1958--2015}, \textit{Geophy. Res. Lett.}, \textbf{44}(9), 4214--4223.

\item
\textbf{{Long}, M.~C.}, C.~A. {Deutsch}, and T.~Ito (2016), Finding forced trends in oceanic oxygen. \textit{Global Biogeochem. Cycles}, \textbf{30}, 381-397.

\item
\textbf{Long, M.~C.}, K.~Lindsay, S.~Peacock, J.~K. Moore, S.~C. Doney (2013), {Twentieth-Century oceanic carbon uptake and storage in CESM1(BGC)}. \textit{J. Clim}, \textbf{26}(18), 6775-6800.

\end{enumerate}

\section{Synergistic activities}

\begin{description}[style=multiline,leftmargin=2.8cm,font=\normalfont]
\setlength{\itemsep}{-0.3em}
\item[2020-2022] Co-Chair of the NCAR Scientists’ Assembly Executive Committee
\item[2020-2023] Member: NOAA Marine Ecosystem Task Force
\item[2019] Lead organizer of the CLIVAR/OCB CMIP6 Hackathon
\item[2018-2020] Member: Ocean Carbon \& Biogeochemistry Scientific Steering Committee
\item[2013] Lead organizer of the 2013 NCAR Advanced Study Program Graduate Student
Colloquium: Carbon-climate connections in the Earth System
\end{description}

\end{document}

%!TEX program = pdflatex
\documentclass[12pt]{article}
\usepackage[left=2.54cm,top=2.54cm,right=2.54cm,bottom=2.54cm]{geometry}
\usepackage[sort&compress,square,numbers]{natbib}
\usepackage{amsmath,url}
\usepackage{graphicx,color,soul}
\usepackage[usenames,dvipsnames]{xcolor}
\usepackage[version=4]{mhchem}
\usepackage{atbegshi}
\usepackage{lipsum}

\usepackage[final]{pdfpages}

%-- fonts
\usepackage{sansmathfonts}
\usepackage[T1]{fontenc}
\usepackage{palatino}
%\usepackage{helvet}
%\renewcommand\familydefault{\sfdefault}
%\renewcommand{\rmdefault}{\sfdefault}

\usepackage[math]{blindtext}
\usepackage[font=footnotesize,labelfont=bf]{caption}

%\usepackage[scale=0.5]{tgpagella}

\urlstyle{rm}

%-- title spacing and style
\usepackage{titlesec}
\titleformat{\part}
{\normalfont\large\bfseries}{\thepart}{1em}{}

\titleformat{\section}
{\normalfont\normalsize\bfseries}{\thesection}{1em}{}
\titleformat{\subsection}
{\normalfont\normalsize\bfseries}{\thesubsection}{1em}{}
\titleformat{\subsubsection}
{\normalfont\normalsize\bfseries}{\thesubsubsection}{1em}{}
\titleformat{\paragraph}[runin]
{\normalfont\normalsize\bfseries}{\theparagraph}{1em}{}
\titleformat{\subparagraph}[runin]
{\normalfont\normalsize\bfseries}{\thesubparagraph}{1em}{}

\titlespacing\part{0pt}{4pt plus 0.1pt minus 0.1pt}{8pt plus 2pt minus 2pt}
\titlespacing\section{0pt}{4pt plus 2pt minus 2pt}{4pt plus 2pt minus 2pt}
\titlespacing\subsection{0pt}{6pt plus 2pt minus 2pt}{4pt plus 2pt minus 2pt}
\titlespacing\subsubsection{0pt}{2pt plus 2pt minus 2pt}{2pt plus 2pt minus 2pt}
\titlespacing\paragraph{0pt}{6pt plus 2pt minus 2pt}{2pt plus 2pt minus 2pt}

\renewcommand{\thesection}{(\alph{section})}

%-- figure
\usepackage{wrapfig}
\usepackage{subcaption}
\usepackage{multirow}
\usepackage{array,booktabs}
\usepackage[inline]{enumitem}
\newlist{inline-enumerate}{enumerate*}{1}
\setlist*[inline-enumerate,1]{%
  label=(\roman*),
}

\newcounter{H}
\newcommand{\hypothesis}[1]{\stepcounter{H}%
\smallskip\paragraph{H\arabic{H}: #1.}}

%-- equation spacing
\AtBeginDocument{%
  \addtolength\abovedisplayskip{-0.2\baselineskip}%
  \addtolength\belowdisplayskip{-0.2\baselineskip}%
  \addtolength\abovedisplayshortskip{-0.2\baselineskip}%
  \addtolength\belowdisplayshortskip{-0.2\baselineskip}%
}

%-- list format
\usepackage{enumitem}
\setlist[description]{font=\bfseries\normalsize}
\renewcommand{\descriptionlabel}[1]{#1:}

%-- general formatting
\renewcommand{\refname}{}
\setlength{\bibsep}{2.0pt}

%-- environments

%-- macros
\makeatletter
\providecommand{\doi}[1]{%
  \begingroup
    \let\bibinfo\@secondoftwo
      doi:\discretionary{}{}{}%
      {#1}%
  \endgroup
}
\makeatother

%-------------------------------------------------------------------------------
%-- begin
%-------------------------------------------------------------------------------
\begin{document}
\thispagestyle{empty}

\begin{center}
\noindent
\textbf{Biographical sketch}

\textbf{Matthew C. Long}

\noindent
National Center for Atmospheric Research\\
P.O. Box 3000, Boulder, CO 80307\\
Phone: 303-497-1311; e-mail: mclong@ucar.edu\\
ORCID: 0000-0003-1273-2957
\end{center}
%------------------------------------------------------------------------------------------------
% Undergraduate, graduate and postdoctoral training, provide institution, major/area, degree and year.
\section{Professional preparation}
\begin{tabular}{lllll}
Tufts University	&	Medford, MA 	& Environmental Engineering	& B.S.	& 1998 	\\
Tufts University	&  	Medford, MA		& Environmental Engineering	& M.S.	& 2000	\\
Stanford University	&	Stanford, CA	& Oceanography				& Ph.D.	& 2010
\end{tabular}

%------------------------------------------------------------------------------------------------
% Beginning with the current position list, in chronological order, professional/academic positions with a brief description.
\section{Appointments}
\begin{description}[style=multiline,leftmargin=2.8cm,font=\normalfont]
\setlength{\itemsep}{-0.5em}
\item[2014--present] {Scientist I, II}; Oceanography Section, NCAR.
\item[2012--2014] {Project Scientist I}; Oceanography Section, NCAR.
\item[2010--2012] {Postdoctoral Fellow}; {Advanced Study Program}, NCAR.
\item[2003--2004] {Water Resources Engineer}; {Camp Dresser \& McKee Inc.}, Cambridge, MA.
\item[2000--2002] {High School Physics \& Geography Teacher}; US Peace Corps, Tanzania.
\end{description}

%------------------------------------------------------------------------------------------------
%Provide a list of up to 10 publications most closely related to the proposed project.  For each publication, identify the names of all authors (in the same sequence in which they appear in the publication), the article title, book or journal title, volume number, page numbers, year of publication, and website address if available electronically. Patents, copyrights and software systems developed may be provided in addition to or substituted for publications.
\section{Five most closely related publications
{\footnotesize ($^*$student led; $^\dagger$postdoc led)}}
\begin{enumerate}[leftmargin=1.5em,font=\normalfont]
\setlength{\itemsep}{-0.3em}

\item
\textbf{{Long}, M.~C.}, Moore, J. K., Lindsay, K., Levy, M., Doney, S. C., Luo, J. Y., et al. (2021). Simulations with the Marine Biogeochemistry Library (MARBL). \textit{JAMES}, \textbf{13}, \\e2021MS002647.

\item
\textbf{{Long}, M.~C.}, T. Ito, and C. Deutsch (2019), Oxygen projections for the future, in Ocean deoxygenation:
everyone’s problem. Causes, impacts, consequences and solutions., edited by D. Laffoley and J. Baxter, doi:10.2305/IUCN.CH.2019.13.en.

\item
$^*${Krumhardt}, K.~M., N.~S. Lovenduski, \textbf{M.~C. Long}, and K.~Lindsay (2017),
  Avoidable impacts of ocean warming on marine primary production: Insights
  from the CESM ensembles, \textit{Global Biogeochem. Cycles},
  \textbf{31}(1), 114--133.

\item
\textbf{{Long}, M.~C.}, C.~A. {Deutsch}, and T.~Ito (2016), Finding forced trends in oceanic oxygen. \textit{Global Biogeochem. Cycles}, \textbf{30}, 381-397.

\item
\textbf{Long, M.~C.}, K.~Lindsay, S.~Peacock, J.~K. Moore, S.~C. Doney (2013), {Twentieth-Century oceanic carbon uptake and storage in CESM1(BGC)}. \textit{J. Clim}, \textbf{26}(18), 6775-6800.

\end{enumerate}

\section{Five other significant publications}
\begin{enumerate}[leftmargin=1.5em,font=\normalfont]
\setlength{\itemsep}{-0.3em}

\item
\textbf{{Long}, M.~C.}, B. B. Stephens, K. McKain, C. Sweeny, R. Keeling, E. A. Kort, et al.
%E. Morgan, J. D. Bent, N. Chandra, F. Chevallier, R. Commane, B. Daube, P. B. Krummel, Z. Loh, I. T. Luijkx, D. Munro, P. Patra, W. Peters, M. Ramonet, C. Rödenbeck, A. Stavert, P. Tans, and S. C Wofsy
(2021), Strong Southern Ocean carbon uptake evident in airborne observations, \textit{Science},
\textbf{374}(6572), {1275-1280}.

\item
Ito, T., \textbf{M.~C. Long}, C.~Deutsch, S.~Minobe, D.~Sun (2019), Mechanisms of low-frequency O$_2$ variability in the North Pacific, \textit{Global Biogeochem. Cycles}, \textbf{33}(2), 110--124.

\item
Ito, T., S.~Minobe, \textbf{M.~C. Long}, C.~Deutsch (2017), {Upper Ocean O$_2$ trends: 1958--2015}, \textit{Geophy. Res. Lett.}, \textbf{44}(9), 4214--4223.

\item
Moore, J.~K., W.~Fu, F.~Primeau, G.~L. Britten, K.~Lindsay, \textbf{M.~C. Long},
%et al.
S.~C.Doney, N.~Mahowald, F.~Hoffman, J.~T. Randerson
(2018), Sustained climate warming drives declining marine biological productivity, \textit{Science}, \textbf{359}(6380), 1139--1143.

\item
$^\dagger$Harrison, C., \textbf{M.~C. Long}, N.~{Lovenduski}, J. K. Moore (2018), {Mesoscale effects on carbon export: a global perspective}. \textit{Global Biogeochem. Cycles}, \textbf{32}(4), 680--703.


\end{enumerate}
\end{document}

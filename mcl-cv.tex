\documentclass[11pt]{article}
\usepackage{charter}
\usepackage{fullpage}

\usepackage{url}
\urlstyle{rm}
\usepackage[hidelinks]{hyperref}

\usepackage{multibib}
\usepackage[nodayofweek]{datetime}
\newdateformat{mydate}{\it\THEDAY{ }\monthname[\THEMONTH] \THEYEAR}

\newcommand{\doi}[1]{\href{http://dx.doi.org/#1}{doi:#1}}
\newcommand{\publink}[1]{}

%-- section style
\usepackage{sectsty}
\sectionfont{\large}
\subsectionfont{\normalsize}
\subsubsectionfont{\normalsize}
\paragraphfont{\normalsize}
\renewcommand{\refname}{}

%-- title spacing
\usepackage{titlesec}
\titlespacing\section{0pt}{12pt plus 2pt minus 2pt}{6pt plus 2pt minus 2pt}
\titlespacing\subsection{0pt}{6pt plus 2pt minus 2pt}{4pt plus 2pt minus 2pt}
\titlespacing\subsubsection{0pt}{6pt plus 2pt minus 2pt}{-4pt plus 2pt minus 2pt}
\titlespacing\paragraph{0pt}{0pt plus 0pt minus 0pt}{0.2em plus 0.25em minus 0em}

%-- list spacing/style
\usepackage{enumitem}
%\setlist{nolistsep}
\setlength{\itemsep}{1em}
\renewcommand{\labelitemi}{$-$}

%-- formatting
\parindent0em

\begin{document}

{\Large \textbf{Matthew C. Long}}
\smallskip

{\large \bf \itshape Curriculum Vitae}

\section*{Contact}
\begin{description}[style=multiline,leftmargin=2.5cm,font=\normalfont]
\item National Center for Atmospheric Research
\hfill 303.497.1311 (\textit{ph})\\
1850 Table Mesa Drive
\hfill 303.497.1700 (\textit{fx})\\
Boulder, Colorado, 80305
\hfill mclong@ucar.edu\\
\href{http://cgd.ucar.edu/staff/mclong}{cgd.ucar.edu/staff/mclong}
\hfill ORCID:
\href{http://orcid.org/0000-0003-1273-2957}{0000-0003-1273-2957} \\
\href{http://github.com/matt-long}{github.com/matt-long}
\hfill ResearcherID:
\href{http://www.researcherid.com/rid/H-4632-2016}{H-4632-2016}
\end{description}


\section{Educational information}

\begin{description}[style=multiline,leftmargin=2.5cm,font=\normalfont]
\item[2010]	\textbf{Ph.D.,  Oceanography}, Stanford University, Stanford, CA
\item[2000]	\textbf{M.S., Environmental Engineering}, Tufts University, Medford, MA
\item[1998]	\textbf{B.S., Environmental Engineering}, Tufts University, Medford, MA
\end{description}


\section{Professional Experience}

\begin{description}[style=multiline,leftmargin=2.5cm,font=\normalfont]
\item[2014--present] \textbf{Scientist I, II, III(tenure-equivalent)},
	{National Center for Atmospheric Research}, \\
	{Climate and Global Dynamics Laboratory},
	{Oceanography Section}

\item[2012--2014] \textbf{Project Scientist I},
	{National Center for Atmospheric Research}, \\
	{Climate and Global Dynamics Division},
	{Oceanography Section}

\item[2010--2012] \textbf{Postdoctoral Fellow},
	{National Center for Atmospheric Research}, \\
	{Advanced Study Program}, {Climate and Global Dynamics Division}

\item[2005--2010] \textbf{Research Assistant}, Stanford University\\
	Developed computer-automated instruments to measure
	inorganic carbon, alkalinity and pH in seawater.
	Operated and maintained Finnigan MAT 252 isotope ratio mass spectrometer
	with Kiel carbonate device and Finnigan MAT Delta+ with Carlo Erba elemental analyzer.

\item[2004--2009] \textbf{Teaching Assistant}, Stanford University \\
	Courses: Introduction to Geology,\\
	\href{http://stanford.sea.edu/}{Stanford at SEA},
	Coastal Oceanography, Antarctic Marine Geology and Geophysics,
	Advanced Oceanography, Oceanic Fluid Dynamics.
	Led \href{http://esw.stanford.edu}{Engineers for a Sustainable World} course to
	design an energy-efficient secondary school in
	\href{http://g.co/maps/d9q9a}{Iringa, Tanzania}.

\item[2003--2004] \textbf{Water Resources Engineer},
	{Camp Dresser \& McKee Inc.}, Cambridge, MA\\
	Developed hydrologic, hydraulic, and water quality models for
	management and system optimization of sewer networks and urban rivers.

\item[2003] \textbf{Field and Laboratory Technician},
	{Desert Research Institute}, Reno, NV
	Species diversity surveys of freshwater springs in Mohave National Preserve. \\
	Surface and ground water quality sampling and analysis on the Truckee River.

\item[2000--2002] \textbf{High School Physics \& Geography Teacher},
	{US Peace Corps, Tanzania}, Ashira Girls Secondary School, {Marangu, Tanzania}
	\href{http://www.peacecorps.gov/index.cfm?shell=learn.wherepc.africa.tanzania}
	{US Peace Corps, Tanzania}\\
	Ashira Girls Secondary School,
	\href{http://g.co/maps/nmzu7}{Marangu, Tanzania}\\
	Taught topics in physical science, weather and climate, \& economic development.
	Wrote a computer manual and taught computer literacy.
	Led a student (16 girls) climb of Mt. Kilimanjaro (5,895 m);
	taught teachers to teach an HIV/AIDS curriculum;
	co-organized a nationwide review of
	the national science and math curriculum.

\item[1998--2000] \textbf{Teaching Assistant}, Tufts University,
	Dept of Civil and Env. Engineering
	Managed environmental engineering teaching laboratory.
	Taught analytical methods, statistical experimental design,
	data analysis and interpretation.


\item[1999] \textbf{Environmental Analyst},
	MA Dept of Public Health,\\ Bureau of Env. Health Assessment, Epidemiology Unit\\
	Developed a GIS-based environmental exposure-assessment protocol
	examining the effect of air pollution on the
	prevalence and distribution of pediatric asthma.

\end{description}


\section{Scientific/Technical Accomplishments}

\begin{description}[style=multiline,leftmargin=0.5cm,font=\normalfont]
\item \textbf{Ocean biogeochemistry and marine ecosystems in the Earth system.}
I lead the development of Marine Biogeochemical Library (MARBL), which is the ocean biogeochemistry component used in the Community Earth System Model (CESM).
In addition to promoting robust scientific representations of the ocean carbon cycle and the biological pump, we designed MARBL to be flexible with respect to coupling with multiple ocean models and invoking ecosystem representations spanning a range in complexity.
I have engaged outside collaborators to contribute to MARBL and continue to build community involvement in MARBL development.
We have also recently begun implementation of the Fisheries Size and Functional Type model (FEISTY) in CESM, aiming to establish a basis for prediction of climate-driven variation in fish.


\item \textbf{Aircraft observations to constrain the carbon cycle.}
I collaborate with Britton Stephens (NCAR) to develop the use of aircraft observations as a constraint on the global carbon cycle.
I co-led the O$_2$/N$_2$ Ratio and CO$_2$ Airborne Southern Ocean Study (ORCAS), which performed intensive airborne surveys over the Southern Ocean aboard the NSF/NCAR Gulfstream V aircraft during Jan--Feb 2016.
%High resolution measurements of atmospheric O$_2$, CO$_2$, and related gases were made to constrain air-sea exchange.
%Additional observations included numerous reactive gases and hyperspectral remote sensing (PRISM) of the surface ocean.
%In support of the field campaign, I developed novel configurations of CESM to forecast atmospheric O$_2$ and CO$_2$ distributions.
Following on the success of ORCAS, we obtained funding to support the Southern Ocean Carbon Gas Observatory (SCARGO), which will collect atmospheric CO$_2$ on the LC-130 aircraft servicing McMurdo Station and South Pole Observatory.
In addition to several other publications, I am lead author of a significant paper using data from ORCAS and other aircraft campaigns to constrain Southern Ocean CO$_2$ fluxes.

\item \textbf{Earth System Data Science.} Effective synthesis and analysis of large datasets is a rate-limiting step to advancing Earth system science.
I have been inspired by recent developments in open-source scientific software, notably those identified by the Pangeo community, that provide technical solutions to Big Data geoscience problems and paradigms for large-scale collaboration.
I have been leading an effort to establish a “community of practice” at NCAR/UCAR aiming to improve collaboration on analytics, explore the margins of what’s possible with data, and more effectively grow our capacity in machine learning and artificial intelligence.


\end{description}


\section{Community Service}

\subsection*{Mentoring}

\begin{description}[style=multiline,leftmargin=0.5cm,font=\normalfont]
\item Postdoctoral researchers supervised
	\begin{itemize}
	\item Jesse Vance (2022--)
	\item N. Precious Mongwe (2018--2021; currently a Researcher at Council for Scientific and Industrial Research (CSIR), Cape Town, South Africa)
	\item Kristen Krumhardt (2018--2020; currently an Associate Scientist at NCAR)
	\item Magdalena Carranza (ASP Fellow, 2018--2020; currently at MBARI)
	\item Jessica Luo (2016--2019; currently a Research Oceanographer at GFDL)
	\item Daniel Whitt (2017; currently Research Scientist at NASA, Ames)
	\item Cheryl Harrison (2015--2017; currently Assistant Professor, Univ of Texas)
	\end{itemize}
\item Ph.D. Dissertation committees
	\begin{itemize}
	\item Zephyr Sylvester (PhD expected 2024, CU Boulder, Advisor: C. Brooks): \textit{Title TBD}.
	\item Sebastian Cantarero (PhD 2022, CU Boulder, Advisor: J. Sepulveda): \textit{Microbial Communities and the Biogeochemistry of the Eastern Tropical South Pacific; a Lipidomic Approach in Natural Environments and Mesocosm Experiments}.
	\item Riley Brady (PhD 2020, CU Boulder, Advisor: N. Lovenduski): \textit{The Variable Circulation and Carbonate Chemistry of Ocean Upwelling Systems}.
	\item Sean Ridge (PhD 2020, Columbia Univ, Advisor: Galen McKinley): \textit{Effects of Ocean Circulation on Ocean Anthropogenic Carbon Uptake}.
	\item F. Garrett Boudinot (PhD 2020, CU Boulder, Advisor: J. Sepulveda):
	\textit{Changes in marine ecology and nitrogen cycling during during a Cretaceous Ocean Anoxic Event }.
	\item Tyler Rohr (PhD 2019, MIT/WHOI, Advisor: Scott Doney):
	\textit{Untangling the controls on Southern Ocean phytoplankton ecosystem dynamics}.
	\item Yassir Eddebbar (PhD 2018, Scripps, Advisor: Ralph Keeling):
	\textit{Climate Modulations of Air-Sea Oxygen, Carbon, and Heat Exchange}.
	\end{itemize}
\item Graduate student visitors hosted at NCAR
	\begin{itemize}
	\item Zephyr Sylvester (CU Boulder, Advisor: Cassandra Brooks, Summer 2019):
	  Informed management of Southern Ocean Krill Fisheries.
	\item Mariela Brooks (Scripps, Advisor: Ralph Keeling, Apr 2018):
	Analysis of oceanic {$^{13}$C} in CESM and comparison to ocean time series.
	\item Sean Ridge (Univ. Wisconsin, Advisor: Galen McKinley, May--Aug 2017):
		Analysis of oceanic carbon-climate feedbacks in the Community Earth System Model (CESM).
	\item Elizabeth Asher (Univ. British Columbia, Advisor: Philippe Tortell, Sept 2013--Apr 2014):
		worked on modeling oxidation pathways of dimethyl sulfide in the
		atmospheric chemistry component of the
		Community Earth System Model (CESM).
	\item Simon Yang (ETH, Advisor: Nicolas Gruber, Jun--Jul 2013):
	added nitrogen isotopes to CESM marine biogeochemistry model, investigation of
	climate and anthropogenic controls on N cycling.
	\item Rebecca Asch (Scripps, Advisor: David Checkley, Mar 2013):
	Phenology of phytoplankton blooms in CESM. \\
	%		Asch, R. G., and M. C. Long, 2014:
	%		Changes in Phytoplankton Phenology Detected with the Community Earth System
	%		Model 1.0 (CESM1): Long-term Trends and the Influence of Climate Oscillations,
	%		\textit{Biogeosciences}, in prep.
	\end{itemize}
\end{description}


\subsection*{Professional activities}

\begin{description}[style=multiline,leftmargin=2.5cm,font=\normalfont]
\item [2022--] Co-Chair: Community Earth System Model, Biogeochemistry Working Group

\item [2022] Co-organizer: Ocean Carbon \& Biogeochemistry Workshop: Marine Carbon Dioxide Removal: Essential Science and Problem Solving for Measurement, Reporting, and Verification

\item [2022--] Working Group \#5 Member: UN Decade of Ocean Science for Sustainable Development programme “Ocean Acidification Research for Sustainability” (OARS)

\item[2022--] Expert Advisor: Ocean Visions LauchPad, supporting selected competitors for the \$100M XPRIZE in Carbon Removal

\item [2020--2022] Co-Chair: NCAR Scientist Assembly, Executive Committee

\item[2021--2022] Member: Ocean Visions Expert Working group -- Designing a Framework for Responsible Research: Sinking Marine Biomass for CO$_2$ Removal

\item [2020--2022] Member: NOAA Marine Ecosystem Task Force
\item [2019] Lead organizer: CLIVAR/OCB CMIP6 Hackathon (\url{cmip6hack.github.io})

\item [2019] Member: NCAR Strategic Plan Steering Committee

\item [2018--2020] Member: Ocean Carbon \& Biogeochemistry (OCB) Scientific Steering Committee

\item [2018] Member: NOAA Integrated Ecosystem Assessment (www.noaa.gov/iea) Climate Change Working Group
\item [2015] Member: Steering group and writing team for the
			NASA Ocean Biology and Biogeochemistry Advanced Science Plan and pre-Decadal Survey Report

\item [2012-2015] Member: CLIVAR/OCB Working Group, Oceanic carbon uptake in CMIP5 models

\item[2013]	Lead organizer: 2013 NCAR Advanced Study Program
			Graduate Student Colloquium:
			\textit{Carbon-climate connections in the Earth System} \\
			\url{https://www.cgd.ucar.edu/events/20130729/}

\item[2004--present] Member, American Geophysical Union
\end{description}


\section{Honors and Awards}

\begin{description}[style=multiline,leftmargin=2.5cm,font=\normalfont]
\item[2022--] Associate Editor, \textit{Journal of Advances in Modeling Earth Systems}
\item[2010--2012] NCAR Advanced Study Program Postdoctoral Fellowship
\item[2006] Antarctic Service Medal
\end{description}


\section{Proposals and Grants}
\begin{description}[style=multiline,leftmargin=2.5cm,font=\normalfont]

\item[2021--2024] NASA, 20-ECOF20-0020,
\textit{Hot spots in the ice: revealing the importance of polynyas for sustaining present and future Antarctic marine ecosystems},
A. K. DuVivier (NCAR), C. M. Brooks (CU/Boulder), S. Jenouvrier (WHOI), S. Labrousse (IPSL), \textbf{M.~C. Long} (NCAR) and M.M. Holland (NCAR).

\item[2021-2024] NSF-EarthCube,
\textit{Collaborative Research: EarthCube Data Capabilities: Project Pythia: A Community Learning Resource for Geoscientists},
J. Clyne (NCAR), R. May (Unidata), K. Paul (NCAR), \textbf{M.~C. Long}  (NCAR), B. E. J. Rose (U Albany), K. Tyle (U Albany).

\item[2021--2024] NASA, Interdisciplinary Research in Earth Science,
\textit{Antarctic marine predators in a dynamic climate},
S. Jenouvrier (WHOI), M. M. Holland (NCAR), \textbf{M.~C. Long} (NCAR), H. Lynch (Stony Brook), M. LaRue (University of Canterbury).

\item[2020--2023] NOAA-OAR-CPO,
\textit{Incorporating fish into Earth system predictions},
\textbf{M.~C. Long} (NCAR), C Petrik (TAMU), S. Siedlecki (UCONN), G. Danabasoglu (NCAR), C. Stock (GFDL).

\item[2020--2023] NOAA-OAR-CPO,
\textit{The predictability of oxygen and its metabolic consequences for fisheries on decadal time scales},
S. Siedlecki (UCONN), \textbf{M.~C. Long} (NCAR), C Petrik (TAMU)

\item[2020--2023] NOAA-OAR-CPO,
\textit{Towards the prediction of fisheries on seasonal to multi-annual time scales},
C Petrik (TAMU), S. Siedlecki (UCONN), \textbf{M.~C. Long} (NCAR).

\item[2020--2023] NSF OCE-1948728,
\textit{Collaborative Research: Forced drivers of trends in ocean biogeochemistry: Volcanos and atmospheric carbon dioxide},
G. McKinley (Columbia), N. Lovenduski (CU/Boulder), \textbf{M.~C. Long} (NCAR).

\item[2020--2023] NSF OCE-1948718,
\textit{Collaborative Research: Mesoscale Drivers of Oxygen in the Tropical Pacific}, Y. Eddebbar (Scripps), A. Subramanian (CU/Boulder),
\textbf{M.~C. Long} (NCAR), D. Whitt (NCAR).

\item[2019--2022] NSF OPP-1839218,
\textit{Collaborative Research: Southern Ocean Carbon Gas Observatory (SCARGO)},
B. B. Stephens (NCAR/EOL), \textbf{M.~C. Long} (NCAR/CGD), K. McKain (CU/CIRES).

\item[2017--2020] NSF OCE-1737158,
\textit{Collaborative Research: Combining Theory and Observations to Constrain Global Ocean Deoxygenation},
T. Ito (GT), C. Deutsch (UW), \textbf{M.~C. Long} (NCAR).

\item[2017-2020] NSF OCE-1735846,
\textit{Collaborative Research: Biogeochemical and physical conditioning of Subantarctic Mode Water in the Southern Ocean},
W. Balch (Bigelow Laboratory), N. Bates (BIOS) P. Morton (Florida State),  D. McGillicuddy (WHOI), \textbf{M.~C. Long} (NCAR).

\item[2017-2020] NSF OCE-1658541,
\textit{Collaborative Research: The impact of climate change on the physics and biology of the ocean on scales down to the submesoscale},
K. Richards (UH), F.O. Bryan (NCAR), \textbf{M.~C. Long} (NCAR), A. Thompson (Caltech).

\item[2015--2017] NSF PLR-1501993,
\textit{O$_2$/N$_2$ Ratio and CO$_2$ Airborne Southern Ocean (ORCAS) Study},
B. Stephens (NCAR) and \textbf{M.~C. Long} (NCAR).

\item[2014--2017] DOE/SciDAC, DE-SC0012603,
\textit{A modular biogeochemical modeling suite for next-generation ocean models},
\textbf{M.~C. Long} (NCAR), K. Lindsay (NCAR), M. Vertenstein (NCAR), M. Maltrud (LANL), and T. Ringler (LANL).

\item[2014--2017] DOE/SciDAC, SC0012605,
\textit{Southern Ocean Uptake in the MPAS-Ocean Model},
W.~G. Large (NCAR), \textbf{M.~C. Long} (NCAR), G. Danabasoglu (NCAR), T. Ringler (LANL), J. Edwards (NCAR), M. Levy (NCAR).

\item[2014--2017] NASA 13-TERAQ13-0089,
\textit{Multi-scale biophysical dynamics governing ocean phytoplankton community structure},
S. C. Doney (WHOI), D. Glover (WHOI), M. Kavanaugh (WHOI),
\textbf{M.~C. Long} (NCAR).

\item[2014--2015] NSF Lower Atmospheric Observing Facilities,
\textit{O$_2$/N$_2$ Ratio and CO$_2$ Airborne Southern Ocean (ORCAS) Study},
B. Stephens (NCAR) and \textbf{M.~C. Long} (NCAR).

\item[2013--2014] USDA-NIFA, GRANT11362158,
\textit{Key uncertainties in the global carbon cycle: Perspectives across terrestrial and ocean ecosystems},
\textbf{M.~C. Long} (NCAR), N.~M. Levine (USC), R.~Q. Thomas (VT), G.~A. McKinley (U. Wisc.).

\end{description}

\section{Publication List}

\makeatletter
\renewcommand\@biblabel[1]{#1.}
\makeatother

\subsection*{Thesis}

\vspace{-1em}
\begin{thebibliography}{99}
\bibitem{Long-2010}
\textbf{Long, M.~C.} (2010), \href{https://stacks.stanford.edu/file/druid:bx376fq2382/jua-kali-augmented.pdf}{Upper ocean physical and ecological dynamics in the {Ross Sea, Antarctica}}, Ph.D. thesis, Stanford University.

\bibitem{ms}
\textbf{Long, M.~C.} (1998),
Monitoring and Modeling of Road Salt in Upper Mystic Lake, M.S. thesis,
Tufts University.
\end{thebibliography}

\subsection*{Refereed Journal Articles}
($^\dagger$postdoc; $^*$student-advisee)

\vspace{-1.5em}

\nocite{*}
\bibliographystyle{agu-based-rev-num}
\bibliography{publications}

%\subsection*{Other Refereed Publications}

\subsection*{Journal Articles In Preparation or Submitted}
\vspace{-1em}

\begin{thebibliography}{99}

\bibitem{C}
$^\dagger$Carranza,	M. M., I. Frenger,	A. Di Luca,	C. Zarzycki, Long	M, D. B. Whitt, R. Brady (2022),
Synoptic-scale weather imprints on upper ocean physics and phytoplankton blooms	in the Southern	Ocean,
\textit{Geophysical	Research Letters},
\textbf{in preparation}.

\bibitem{A}
\textbf{M.~C. {Long}}, C. Deutsch, $^\dagger$Mongwe, N. P., and T. Ito (2022), Climatic controls on metabolic constraints in the ocean,
\textit{Nature},
\textbf{in preparation}.

\bibitem{Brett-etal}
$^\dagger$Brett, G.~J., D. B. Whitt, \textbf{M.~C. {Long}}, Frank O. Bryan, Kelvin J. Richards (2022), Submesoscale effects on changes to export production under global warming,
\textit{Journal of Geophysical Research},
\textbf{in preparation}.

\bibitem{Negrete-García-etal}
Negrete-García, G., J.~Y. $^\dagger$Luo, \textbf{M.~C. {Long}}, K. Lindsay, M. Levy, A.~D. Barton (2022), Plankton energy flows using a global size-structured and trait-based model,
\textit{Progress in Oceanography},
\textbf{in press}.


\bibitem{A}
Margolskee, A. J., C. Deutsch, T. Ito, \textbf{M.~C. {Long}} (2022),
Multi-decadal oxygen loss in the North Atlantic amplified by temperature sensitivity of phytoplankton growth,
\textit{Journal of Geophysical Research},
\textbf{in preparation}.


\end{thebibliography}

\subsection*{Non-refereed Publications}

\vspace{-1em}
\begin{thebibliography}{99}

\bibitem{Dunne-Romanou-etal-2019}
Dunne, J., N. Romanou, G. McKinley, \textbf{M.~C. Long}, S. Doney (2019), Synthesis and Intercomparison of Ocean Carbon Uptake in CMIP6 Models Workshop Report. 35~pp; \doi{10.1575/1912/24038}.

\bibitem{Fassbender-Palter-etal-2018}
Fassbender, A. J., J. B. Palter, \textbf{M.~C. Long}, T. Ito, S. P. Bishop, and M. F. Cronin (2018), Ocean Carbon Hot Spots. A Joint US CLIVAR
and OCB Workshop Report, 2018-3, 34~pp., \doi{10.5065/D6Z036ZS}.

\bibitem{Long-2018}
\textbf{Long, M.~C.}, The oceans are gasping for air (2018), editorial, \textit{The Mark News}.

\bibitem{DiNezio-Barbero-etal-2015}
{DiNezio}, P.~N., L.~{Barbero}, \textbf{M.~C. Long}, N.~{Lovenduski}, and C.~Deser (2015), {Anthropogenic changes in the tropical ocean carbon cycle masked by Pacific Decadal Variability?} \textit{US-CLIVAR Variations},  \textbf{13~(2)}.
\publink{DiNezio-Barbero-etal-2015}

\bibitem{Bracco-Long-etal-2015}
Bracco, A., \textbf{M.~C. Long}, N.~M. Levine, R.~Q. Thomas, C.~{Deutsch}, and G.~A. McKinley (2015), {NCAR's Summer Colloquium: Capacity building in Cross-disciplinary Research of Earth System Carbon-climate Connections}. \textit{Bull. Amer. Meteor. Soc.}, \doi{10.1175/BAMS-D-13-00246.1}.
\publink{Bracco-Long-etal-2015}

\bibitem{Thomas-McKinley-etal-2013}
Thomas, R.~Q., G.~A. McKinley, and \textbf{M.~C. Long} (2013), Examining uncertainties in representations of the carbon cycle in Earth system models. \textit{Eos}, \textbf{94~(48)}, 460--460, \doi{10.1002/2013EO480006}.

\bibitem{Long2007}
\textbf{Long, M.~C.} (2007),
Climate driving of marine ecosystem changes: a perspective on
physical-biological coupling. {\em IMBER Update, newsletter of Integrated Marine
Biogeochemistry and Ecosystem Research}.

\end{thebibliography}

\bigskip
\mydate
\rightline{\today}

\end{document}
